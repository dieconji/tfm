\pagenumbering{roman} 
\setcounter{page}{1} 
\pagestyle{plain}

%%%%%%%%%%%%%%%%
%%% CREDITOS %%%
%%%%%%%%%%%%%%%%
\chapter*{Copyright}

\vspace{1cm}

\begin{figure}[ht]
    \centering
	\includegraphics[scale=1]{images/license.png}
\end{figure}
\\
Esta obra está sujeta a una licencia de Reconocimiento -  NoComercial - SinObraDerivada
\\
\href{https://creativecommons.org/licenses/by-nc-nd/3.0/es/}{3.0 España de CreativeCommons}.
\\
Esta obra está sujeta a una licencia de Reconocimiento-NoComercial- SinObraDerivada 3.0 España de Creative Commons
\\
\\
\\
Copyright © 2019 Diego Contreras Jiménez
\\
\\
Reservados todos los derechos. Está prohibida la reproducción total o parcial de esta obra por cualquier medio o procedimiento, comprendidos la impresión, la reprografía, el microfilme, el tratamiento informático o cualquier otro sistema, así como la distribución de ejemplares mediante alquiler y préstamo, sin la autorización escrita del autor o de los límites que autorice la Ley de Propiedad Intelectual.

%%%%%%%%%%%%%
%%% FICHA %%%
%%%%%%%%%%%%%
\chapter*{FICHA DEL TRABAJO FINAL}

\begin{table}[ht]
	\centering{}
	\renewcommand{\arraystretch}{2}

	\begin{tabular}{r | l}
		\hline
		Título del trabajo: & Análisis predictivo del comportamiento de los clientes \\ & en sus interacciones con la empresa \\
		\hline
        Nombre del autor: & Diego Contreras Jiménez\\
		\hline
        Nombre del colaborador docente: & Jordi Nin Guerrero\\
		\hline
        Nombre del PRA: & Jordi Casas Roma\\
		\hline
        Fecha de entrega: & 06/2019\\
		\hline
        Titulación o programa: & Máster universitario de Ciencia de datos\\
		\hline
        Área del Trabajo Final: & Big Data Analytics\\
		\hline
        Idioma del trabajo: & Español\\
		\hline
        Palabras clave & Predictive analytics \\ & User interactions \\ & Customer experience\\
		\hline
	\end{tabular}
\end{table}

%%%%%%%%%%%%%%%%%%%
%%% DEDICATORIA %%%
%%%%%%%%%%%%%%%%%%%
\chapter*{Cita}

``Information is the oil of the 21st century, and analytics is the combustion engine" 

(Peter Sondergaard, Senior VP, Gartner).

%%%%%%%%%%%%%%%%%%%
%%% Agradecimientos %%%
%%%%%%%%%%%%%%%%%%%
\chapter*{Agradecimientos}

Gracias a mi pareja Erika, que ha estado siempre apoyándome y ayudándome en todo lo posible para que pudiera dedicarme a la realización de este trabajo y a la culminación del mismo.
\\
Tampoco puedo dejar de agradecer a mi tutor de máster Jordi Nin Guerrero que, con su esfuerzo y dedicación, sus conocimientos y certera dirección hicieron posible la realización de este trabajo.
\\
\\
A todos, mi más sincera gratitud.
\\
El autor


%%%%%%%%%%%%%%%%
%%% RESUMEN  %%%
%%%%%%%%%%%%%%%%
\chapter*{Abstract}
\addcontentsline{toc}{chapter}{Abstract}

\onehalfspacing

Customers communicate with companies through several ways (e.g., phone, web, mobile applications, social networks, e-mail...). The purpose of these interactions can be for different reasons, for example:
\begin{itemize}
  \item Obtaining information on products and prices.
  \item Consulting invoices.
  \item Making claims for any problem with their service.
  \item Contracting new products or services.
  \item Modifying any terms of existing contracts.
  \item Terminating the contract itself.
\end{itemize}

Companies usually have CRM systems which manage interactions with existing and potential customers. An important feature of these systems is that they allow information to be collected from the different channels.

In the case of large companies with millions of customers, processing all this data can be very complex and slow, but there are massive processing technologies such as parallel computing that allow to reduce time considerably.

The log of user interactions can be analyzed to understand their preferences and needs, so companies can improve business relationships with their customers and therefore, provide a better customer experience.

Artificial Intelligence and Machine Learning algorithms can be used to influence business decision making. For example, predictive analytics can lead to increase cross-selling and customer retention. In addition, all these improvements could have an impact on the reduction of operating costs derived from the impact of telephone calls to the call center. 

\vspace{1.5cm}

\textbf{Keywords}: Predictive analytics, User interactions, Customer experience 

\chapter*{Resumen}
\addcontentsline{toc}{chapter}{Resumen}

\onehalfspacing

Los clientes interactúan con las empresas a través de diversas formas de comunicación (e.g., teléfono, web, aplicaciones móviles, redes sociales, e-mail...). La finalidad de estas comunicaciones puede ser por distintos motivos, como por ejemplo: 

\begin{itemize}
  \item Obtener información de productos y precios.
  \item Consultar la facturación.
  \item Hacer reclamaciones por cualquier problema con su servicio.
  \item Contratar nuevos productos o servicios.
  \item Modificar cualquier condición de los contratos existentes.
  \item Dar de baja el propio contrato.
\end{itemize}

Las empresas suelen disponer de un sistema CRM en donde se gestionan las interacciones con los clientes, tanto los propios como los potenciales. Una funcionalidad importante de estos sistemas es que permiten recopilar información de los distintos canales. 

En el caso de las grandes empresas que cuentan con millones de clientes el procesamiento de todos estos datos puede ser muy complejo y lento, pero existen tecnologías de procesado masivo como la computación paralela que permiten reducir el tiempo considerablemente. 

Todo este historial de interacciones con los usuarios puede ser analizado posteriormente para comprender sus preferencias y necesidades, de modo que las empresas pueden mejorar las relaciones comerciales con sus clientes y por lo tanto ofrecer una mejor experiencia del cliente. 

Así pues, se puede hacer uso de algoritmos de Inteligencia Artificial y Machine Learning para influir en la toma de decisiones de la empresa. Por ejemplo, este análisis predictivo puede provocar un aumento de las ventas cruzadas y de la retención de clientes. Además, todas estas mejoras podrían repercutir en la reducción de los costes operativos derivados del impacto de las llamadas telefónicas al centro de llamadas.

\vspace{1.5cm}

\textbf{Palabras clave}: Analítica predictiva, interacciones de usuario, experiencia de usuario