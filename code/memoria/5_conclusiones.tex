\chapter{Conclusiones}
\label{chapter:conclusiones}

Mi valoración, en general, ha sido muy positiva porque hemos trabajado en el ciclo completo de un proyecto analítico de \textit{``Big Data"} con datos reales de mi actual empresa, y hemos conseguido desarrollar los objetivos parciales planteados en el inicio del proyecto. Así pues, se han realizado dos análisis predictivos clasificatorios distintos:
\begin{itemize}
    \item Predicción de facturación electrónica de los clientes.
    \item Predicción del canal por el que se ha realizado una reclamación.
\end{itemize}

Cabe recalcar que, puesto que el proyecto pertenece al aula \textit{``Big Data Analytics"}, hemos usado únicamente entornos de este tipo para su consecución.
A pesar de encontrarnos con un entorno Azure HDInsight todavía con carencias por estar en una fase temprana de despliegue, hemos podido continuar con el desarrollo en el entorno Databricks Community Edition. Como ambos sistemas son compatibles por usar Apache Spark, cuando estos errores detectados se solventen se podrán migrar todos los modelos sin mayor problema.

\\

Además, la realización del TFM me ha ayudado a profundizar conocimientos vistos en otras asignaturas del máster, así como a adquirir y aprender nuevas habilidades y tecnologías.
Algunas de las materias trabajadas han:
\begin{itemize}
    \item Profundiar el conocimiento del lenguaje de programación Python.
    \item Ampliar el conocimiento en el uso de librerías de \textit{``Machine Learning"} como Apache MLlib \cite{mllib}.
    \item Trabajar con una metodología \textit{``Agile"}.
    \item Aprender el lenguaje LaTeX\cite{latex} para la redacción de la memoria.
\end{itemize}
\\

